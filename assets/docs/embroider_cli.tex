\chapter{The \texttt{embroider} Command Line Program}


\subsection{Embroider pipeline}

Adjectives apply to every following noun so

\begin{lstlisting}
embroider --satin 0.3,0.6 --thickness 2 --circle 10,20,5 \
    --border 3 --disc 30,40,10 --arc 30,50,10,60 output.pes
\end{lstlisting}

Creates:

\begin{itemize}
\item a circle with properties: thickness 2, satin 0.3,0.6
\item a disc with properties: 
\item an arc with properties:
\end{itemize}

in that order then writes them to the output file `output.pes`.

\subsection{embroider CLI}

\begin{itemize}
\item Make `-circle` flag to add a circle to the current pattern.
\item Make `-rect` flag to add a rectangle to the current pattern.
\item Make `-fill` flag to set the current satin fill algorithm for the current geometry. (for example `-fill crosses -circle 11,13,10` fills a circle with center 11mm, 13mm with radius 10mm with crosses).
\item Make `-ellipse` flag to add to ellipse to the current pattern.
\item Make `-bezier` flag to add a bezier curve to the current pattern.
\end{itemize}
