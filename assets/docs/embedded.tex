\chapter{Embroiderbot and Libembroidery on Embedded Systems}

The libembroidery library is designed to support embedded environments, so it can be used in CNC applications.

\section{Compatible Boards}

We recommend using an Arduino\index{Arduino} Mega 2560 or another board with equal or
greater specs. That being said, we have had success using an Arduino Uno
R3 but this will likely require further optimization and other
improvements to ensure continued compatibility with the Uno. See below
for more information.

\section{Arduino Considerations}

There are two main concerns here: Flash Storage and SRAM.

libembroidery continually outgrows the 32KB of Flash storage on the Arduino Uno and every time this occurs, a decision has to be made as to what capabilities should be included or omitted. While reading files is the main focus on arduino, writing files may also play a bigger role in the future. Long term, it would be most practical to handle the inclusion or omission of any feature via a single configuration header file that the user can modify to suit their needs.

SRAM is in extremely limited supply and it will deplete quickly so any dynamic allocation should occur early during the setup phase of the sketch and sparingly or not at all later in the sketch. To help minimize SRAM consumption on Arduino and ensure libembroidery can be used in any way the sketch creator desires, it is required that any sketch using libembroidery must implement event handlers. See the ino-event source and header files for more information.

There is also an excellent article by Bill Earl on the Adafruit Learning
System which covers these topics in more depth:
[http://learn.adafruit.com/memories-of-an-arduino?view=all](http://learn.adafruit.com/memories-of-an-arduino?view=all).

\section{Space}

Since a stitch takes 3 bytes of storage and many patterns use more than
10k stitches, we can't assume that the pattern will fit in memory. Therefore
we will need to buffer the current pattern on and off storage in small
chunks. By the same reasoning, we can't load all of one struct beore
looping so we will need functions similar to binaryReadInt16 for each
struct.

This means the EmbArray approach won't work since we need to load
each element and dynamic memory management is unnecessary because
the arrays lie in storage.

TODO: Replace EmbArray functions with embPattern load functions.

\section{Tables}

All thread tables and large text blocks are too big to compile directly
into the source code. Instead we can package the library with a data packet
that is compiled from an assembly program in raw format so the specific
padding can be controlled.

In the user section above we will make it clear that this file
needs to be loaded on the pattern USB/SD card or the program won't function.

TODO: Start file with a list of offsets to data with a corresponding table
to load into with macro constants for each label needed.

\section{Current Pattern Memory Management}

It will be simpler to make one file per EmbArray so we keep an EmbFile*
and a length, so no malloc call is necessary. So there needs to be a consistent
tmpfile naming scheme.

TODO: For each pattern generate a random string of hexadecimal and append it
to the filenames like `stitchList\_A16F.dat`. Need to check for a file
which indicates that this string has been used already.

\section{Special Notes}

Due to historical reasons and to remain compatible with the Arduino 1.0
IDE, this folder must be called ``utility''. Refer to the arduino build
process for more info:
[https://arduino.github.io/arduino-cli/0.19/sketch-build-process/](https://arduino.github.io/arduino-cli/0.19/sketch-build-process/).

libembroidery relies on the Arduino SD library for reading files. See
the ino-file source and header files for more information.

\section{The Assembly Split}

One problem to the problem of supporting both systems with abundant memory
(such as a 2010s or later desktop) and with scarce memory (such as embedded
systems) is that they don't share the same assembly language. To deal with
this: there will be two equivalent software which are hand engineered to be
similar but one will be in C and the other in the assembly dialects we support.

All assembly will be intended for embedded systems only, since a slightly
smaller set of features will be supported. However, we will write a
`x86` version since that can be tested.

That way the work that has been done to simplify the C code can be applied to
the assembly versions.
