\usepackage{amsmath}
\usepackage{listings}
\usepackage{hyperref}
\usepackage{multicol}
\usepackage{graphicx}

\lstset{
  basicstyle=\footnotesize\ttfamily
}

% Version Info
% ------------
%
% * MAJOR.txt - Contains the major release version number.
% This will be updated to 2 when Embroidermodder 2 is final.
% Until then, it is set to 1 while Embroidermodder 2 is in development.
%
% * MINOR.txt - Contains the minor release version number.
% This file will be updated frequently when minor releases happen.
% It will be set to zero when Embroidermodder 2 is final.
% Until then, it is set to 90+ while Embroidermodder 2 is in development.
% The number shall be 2 digits, padded with a zero for single digit numbers.
%
% * PATCH.txt - Contains the patch release version number.
% This file will always be zero. Nightly builds will not use this file and
% will use the build unix date as such: ```date +%Y%m%d%H%M%S```
% This ensures that the latest build is always the highest number.

\newcommand{\emversion}{1.90.0}
\newcommand{\libembversion}{1.0.0}

\title{Embroidermodder User Manual}
\author{The Embroidermodder Team}

\begin{document}
\maketitle

\tableofcontents

\section{The Embroidermodder Project and Team}

The \emph{Embroidermodder 2} project is a collection of small software utilities for
manipulating, converting and creating embroidery files in all major embroidery
machine formats. The program \emph{Embroidermodder 2} itself is a larger graphical
user interface (GUI) which is at the heart of the project.

The tools and associated documents are:

\begin{itemize}
\item The website (www.libembroidery.org), which is maintained [docs-repo]_.
\item The manual [manual-link]_ covering all these projects which is maintained seperately as LaTeX [manual-src]_.
\item The GUI (embroidermodder), maintained \texttt{Embroidermodder 2}.
\item The core library of low-level functions: \texttt{libembroidery}.
\item The CLI `embroider` which is part of [libembroidery-src]_.
\item Mobile embroidery format viewers and tools [EmbroideryMobile-src]_.
\item Specs for an open hardware embroidery machine called Embroiderbot (not started yet) which is also part of [libembroidery-src]_.
\end{itemize}

.. [docs-repo] https://github.com/Embroidermodder/docs
.. [manual-link] https://www.libembroidery.org/embroidermodder_2.0_manual.pdf
.. [manual-src] https://github.com/Embroidermodder/emrm
.. [gui] https://github.com/Embroidermodder/embroidermodder
.. [libembroidery-src] https://github.com/Embroidermodder/libembroidery
.. [EmbroideryMobile-src] https://github.com/Embroidermodder/embroiderymobile

They are all tools to make the standard user experience of working with an
embroidery machine better without expensive software which is locked to specific
manufacturers and formats. But ultimately we hope that the core *Embroidermodder 2*
is a practical, ever-present tool in larger workshops, small cottage industry workshops
and personal hobbyist's bedrooms.

Embroidermodder 2 is licensed under the zlib license and we aim to keep all of
our tools open source and free of charge. If you would like to support the
project check out our `Open Collective <https://opencollective.com/embroidermodder>`_
group. If you would like to help, please
join us on GitHub. This document is written as developer training as well
helping new users (see the last sections) so this is the place to learn how
to start changing the code.

The Embroidermodder Team is the collection of people who've submitted
patches, artwork and documentation to our three projects.
The team was established by Jonathan Greig and Josh Varga.
The full list of contributors who wish to be credited is
here: https://www.libembroidery.org/docs/credits/

\subsection{Core Development Team}

Embroidermodder 2:

\begin{itemize}
\item Jonathan Greig \url{https://github.com/redteam316}
\item Josh Varga \url{https://github.com/JoshVarga}
\item Robin Swift [https://github.com/robin-swift](https://github.com/robin-swift)
\end{itemize}

Embroidermodder 1:

  * Josh Varga [https://github.com/JoshVarga](https://github.com/JoshVarga)
  * Mark Pontius [http://sourceforge.net/u/mpontius/profile](http://sourceforge.net/u/mpontius/profile)

## History

Embroidermodder 1 was started by Mark Pontius in 2004 while staying up all night
with his son in his first couple months. When Mark returned to his day job,
he lacked the time to continue the project. Mark made the decision to focus on his
family and work, and in 2005, Mark gave full control of the project to Josh Varga
so that Embroidermodder could continue its growth.

Embroidermodder 2 was conceived in mid 2011 when Jonathan Greig and Josh Varga
discussed the possibility of making a cross-platform version. It is currently in
active development and will run on GNU/Linux, Mac OS X, Microsoft Windows and Raspberry Pi.

The source code and binaries for Embroidermodder 1 were hosted on Sourceforge, but
due to link rot we've lost them.

!!! todo
    upload a backup here.

The source code for Embroidermodder
2 was moved to GitHub (\url{https://github.com/Embroidermodder/Embroidermodder})
on July 18, 2013.

This website was moved to
GitHub (\url{https://github.com/Embroidermodder/website}) on September 9, 2013. Due to us losing the domain name it was renamed to
\texttt{website} from \texttt{www.embroidermodder.org} and the new url \url{https://www.libembroidery.org}.

The libembroidery library (https://github.com/Embroidermodder/libembroidery)
became a seperate project in 2018 as a way of supporting other frontends with the
same file parsing and geometry routines.

\subsection{History}

Embroidermodder 1 was started by Mark Pontius in 2004 while staying up all night
with his son in his first couple months. When Mark returned to his day job, he
lacked the time to continue the project. Mark made the decision to focus on his
family and work, and in 2005, Mark gave full control of the project to Josh Varga
so that Embroidermodder could continue its growth.

Embroidermodder 2 was conceived in mid 2011 when Jonathan Greig and Josh Varga
discussed the possibility of making a cross-platform version. It is currently in
active development and will run on GNU/Linux, Mac OS X, Microsoft Windows and
Raspberry Pi.

The source code and binaries for Embroidermodder 1 were hosted on Sourceforge, but
due to link rot we've lost them.

!!! warning

    TODO: upload a backup here.

The `source code for Embroidermodder 2 <https://github.com/Embroidermodder/Embroidermodder>`_
was moved to GitHub on July 18, 2013.

`This website <https://github.com/Embroidermodder/docs>`_ was moved
to GitHub on September 9, 2013. Due to us losing the domain name it was renamed
to ``www.libembroidery.org`` from ``www.embroidermodder.org``.

The `libembroidery library <https://github.com/Embroidermodder/libembroidery>`_
became a seperate project in 2018 as a way of supporting other frontends with
the same file parsing and geometry routines.

\section{Batch Conversion}

\texttt{WARNING: THIS FEATURE IS NOT FUNCTIONAL, THIS SECTION IS PLANNING.}

Many users of \texttt{embroider} will only want our batch conversion feature.

\section{Changelog}

\subsection{From early alpha to beta}

\begin{itemize}
\item Up and Down keys cycle thru commands in the command prompt.
\end{itemize}

\section{Contributing}

\subsection{To Do}

\begin{itemize}
\item Copy of license in appendix.
\item Copy of manpage in appendix.
\end{itemize}

\begin{verbatim}
```
EMBROIDER(1)                 General Commands Manual                EMBROIDER(1)

NAME
       embroider - a command line program for machine embroidery

SYNOPSIS
       Copyright 2013-2023 The Embroidermodder Team Licensed under the terms of
       the zlib license.

       https://github.com/Embroidermodder/libembroidery
       https://www.libembroidery.org

       Usage: embroider [OPTIONS] fileToRead...

       Conversion:
           -t, --to        Convert all files given to the format specified
                           by the arguments to the flag, for example:
                               $ embroider -t dst input.pes
                           would convert
                           in the same directory the program runs in.

                           The accepted input formats are (TO BE DETERMINED).
                           The accepted output formats are (TO BE DETERMINED).

       Output:
           -h, --help       Print this message.
           -F, --formats     Print help on the formats that embroider can deal
       with.
           -q, --quiet      Only print fatal errors.
           -V, --verbose    Print everything that has reporting.
           -v, --version    Print the version.

       Modify patterns:
           --combine        takes 3 arguments and combines the first
                            two by placing them atop each other and
                            outputs to the third
                               $ embroider --combine a.dst b.dst output.dst

       Graphics:
           -c, --circle     Add a circle defined by the arguments given to the
       current pattern.
           -e, --ellipse    Add a circle defined by the arguments given to the
       current pattern.
           -l, --line       Add a line defined by the arguments given to the
       current pattern.
           -P, --polyline   Add a polyline.
           -p, --polygon    Add a polygon.
           -r, --render     Create an image in PNG format of what the embroidery
       should look like.
           -s, --satin      Fill the current geometry with satin stitches
       according
                            to the defined algorithm.
           -S, --stitch     Add a stitch defined by the arguments given to the
       current pattern.

       Quality Assurance:
               --test       Run the basic test suite.
               --full-test-suite  Run all tests, even those we expect to fail.

                                                                    EMBROIDER(1)
\end{verbatim}


\begin{verbatim}
EMBROIDER
    A command line program for machine embroidery.
    Copyright 2013-2021 The Embroidermodder Team
    Licensed under the terms of the zlib license.

    https://github.com/Embroidermodder/libembroidery
    https://embroidermodder.org

Usage: embroider [OPTIONS] fileToRead...

Conversion:
-t, -to         Convert all files given to the format specified
                by the arguments to the flag, for example:
                    $ embroider -t dst input.pes
                would convert \``input.pes\`` to \``input.dst\``
                in the same directory the program runs in.

                The accepted input formats are (TO BE DETERMINED).
                The accepted output formats are (TO BE DETERMINED).

Output:
-h, -help       Print this message.
-f, -format     Print help on the formats that
                embroider can deal with.
-q, -quiet      Only print fatal errors.
-V, -verbose    Print everything that has reporting.
-v, -version    Print the version.

Graphics:
-c, -circle     Add a circle defined by the arguments
                given to the current pattern.
-e, -ellipse    Add a circle defined by the arguments
                given to the current pattern.
-l, -line       Add a line defined by the arguments
                given to the current pattern.
-P, -polyline   Add a polyline.
-p, -polygon    Add a polygon.
-s, -satin      Fill the current geometry with satin
                stitches according
                to the defined algorithm.
-S, -stitch     Add a stitch defined by the arguments
                given to the current pattern.

Quality Assurance:
    -test       Run the test suite.# Full Colour Photograph, Vector and Black and White Designs

\end{verbatim}

WARNING: THIS FEATURE IS NOT AT ALL PRESENT, THIS SECTION IS PLANNING.

If you are starting an embroidery from an image, it's important to note that \embname uses very different
approaches based on what kind of image is fed into it. A photograph is the most difficult starting point
for an automated system since there are many artistic decisions which are mutually exclusive and
no software should make those decisions for you. At least, you should be aware that a decision is being made.
On the other hand, a vector based design gives the program the best starting point while still not being
a machine embroidery file. Finally a scan of a vector image or hand-drawn inkwork is somewhere in-between
these options and is the recommended option for people who work best on paper.

The following subsections are in order of how good the final results should be.

## Vector Designs
\index{vector}


## Scans of Hand-drawn Designs in a Limited Palette
\index{scans}


## Scans of Hand-drawn Designs in Block Colours
\index{scans}


## Full Colour Photographs
\index{photo}



\subsection{Command Line Generation}


\section{Embroidermodder on Mobile}

\section{Portable Embroidery Tool}

\section{Online Viewer and Converter}

!!! warning
    EXPERIMENTAL

\begin{verbatim}
<script>
  /* Call clang generated WASM here. */
  
</script>

If you only need to convert and view machine embroidery files (like our old Android application) then this page
does just that. To access other features of the Embroidermodder Project please see the [downloads.html](Downloads page).

<!-- Uses the native file dialog to get the string as a file object that is passed to a function from script above.
     If this is not called first, produce an error message. -->
<button onclick="upload();">Upload File</button>

<!-- Displays the SVG file output as a widget below. This could be by default. -->
<button onclick="show_svg();">Show</button>

<!-- Brings up the native file dialog, call "convert" with the arguments. -->
<button onclick="export_to();">Export...</button>

<svg class="viewer"></svg>
\end{verbatim}

This viewer uses no cookies and no external tools, so if you save this webpage to use offline it will still function.
Eventually, this webpage can be embedded in both an Android and an iOS app so we have, in total, 3 front-ends: embroider,
embroidermodder and the viewer/converter.

\section{Creating a Design}

WARNING: THIS FEATURE IS NOT FUNCTIONAL, THIS SECTION IS PLANNING.
# Credits for Embroidermodder 2, libembroidery and all other related code

Please note that this file in not in alphabetical order. If you have
contributed and wish to be added to this list, create a new credit
bullet. Fill it in with your information and submit it to us. Supply
your: your full name or pseudonym and GitHub handle, if available.

Kinds of contribution:

\begin{itemize}
\item Documentation - for changes to README files, manuals or help files.
\item Artwork - for artwork other than designs.
\item Bug Fixes - for small patches of a few lines.
\item Translation - for large patches to the translation files.
\item Designs - for an embroidery design sample or parametrized design as a toml file.
\item Bindings - for programming language bindings for libembroidery.
\item Commands - for Embroidermodder 2's in-built terminal.
\end{itemize}

finally there's \textbf{Core Developer} which is reserved for long term
contributors.

\subsection{Contributors}

!!! warning
    Need to fix script to generate from ``data``.

\section{Editing Designs}

WARNING: THIS FEATURE IS NOT FUNCTIONAL, THIS SECTION IS PLANNING.

\subsection{Command Line Editing}

\section{Introduction}

title: The Embroidermodder Project
description: Free and Open Source Software for Machine Embroidery.
keywords: machine embroidery, embroidery, dst, pes, jef

WARNING  ( IN ALPHA DEVELOPMENT: NOT YET READY FOR SERIOUS USE. )

Embroidermodder is a free machine embroidery software program.
The newest version, Embroidermodder 2 can:

- edit and create embroidery designs
- estimate the amount of thread and machine time needed to stitch a design
- convert embroidery files to a variety of formats
- upscale or downscale designs
- run on Windows, Mac and Linux

For more in-depth information, see [our website](http://www.libembroidery.org)
and get the manuals [here](http://www.libembroidery.org/documentation).

To try out the software in alpha see our current
[alpha pre-release](https://github.com/Embroidermodder/Embroidermodder/releases).

Various sample embroidery design files can be found in
the src/samples folder.

Embroidermodder is developed by The Embroidermodder Team which is maintained as a
list on the website under ["Credits"](http://www.libembroidery.org/credits).

## Screenshots

If you use multiple operating systems, it's important to choose software that works on all of them.

Embroidermodder 2 runs on Windows, Linux and Mac OS X. Let's not forget the [Raspberry
Pi](http://www.raspberrypi.org).

![features: platforms 1](images/features-platforms-1.png)

### Realistic Rendering

(This feature is currently broken.)

It is important to be able to visualize what a design will look like when stitched and our
pseudo ``3D'' realistic rendering helps achieve this.

Realistic rendering sample \#1:

![features real render 1](images/features-realrender-1.png)

Realistic rendering sample \#2:

![features real render 2](images/features-realrender-2.png)

Realistic rendering sample \#3:

![features real render 3](images/features-realrender-3.png)

Various grid types and auto-adjusting rulers

Making use of the automatically adjusting ruler in conjunction with the grid will ensure your
design is properly sized and fits within your embroidery hoop area.

Use rectangular, circular or isometric grids to construct your masterpiece!

Multiple grids and rulers in action:

![features grid ruler](images/features-grid-ruler-1.png)

### Many measurement tools

Taking measurements is a critical part of creating great designs. Whether you are designing
mission critical embroidered space suits for NASA or some other far out design for your next
meet-up, you will have precise measurement tools at your command to make it happen. You can
locate individual points or find distances between any 2 points anywhere in the design!

Take quick and accurate measurements:

![features measure 1](images/features-measure-1.png)

### Add text to any design

Need to make company apparel for all of your employees with individual names on them? No sweat.
Just simply add text to your existing design or create one from scratch, quickly and easily.
Didn't get it the right size or made a typo? No problem. Just select the text and update it
with the property editor.

Add text and adjust its properties quickly:

![features text 1](images/features-text-1.png)

### Supports many formats

Embroidery machines all accept different formats. There are so many formats available that it
can sometimes be confusing whether a design will work with your machine.

Embroidermodder 2 supports a wide variety of embroidery formats as well as several vector
formats, such as SVG and DXF. This allows you to worry less about which designs you can use.

\subsubsection{Batch Conversion}

(Currently this being ported to the \texttt{embroider} command line program.)

Need to send a client several different formats? Just use libembroidery-convert, our command
line utility which supports batch file conversion.

There are a multitude of formats to choose from:

\begin{figure}[H]
\centering}
\includegraphics[width=0.8\textwidth]{images/features-formats-1.png}
\caption{features formats}
\end{figure}

\subsubsection{Scripting API}

The GUI works by emitting internal text commands, so if you want to alter
or add features to the program that aren't as low level as these commands then you
can chain them together in simple scripts. This allows more control over the program than
the GUI can offer.

A (no longer current) Embroidermodder 2 command excerpt:

\begin{figure}[H]
\centering
\includegraphics[width=0.8\textwidth]{images/features-scripting-1.png}
\caption{scripting screenshot}
\end{figure}

\subsection{Dependencies}

To build Embroidermodder 2 from source you will need at least
[the Embroidermodder 2 source code itself](https://github.com/Embroidermodder/Embroidermodder),
a build environment including [CMake](https://cmake.org) and [Qt](http://www.qt-project.org) (version >= 6.0). For advice on how to get these,
see the following subsections.

You will also need the git submodules, which can be collected by running these lines
from the embroidermodder source directory:

```
git submodule init
git submodule update
```

### Debian/Ubuntu repository packages

The Qt, KDE and Valgrind build dependencies can be installed easily by
opening a terminal and issuing these commands:

```
sudo apt-get update
sudo apt-get install cmake build-essential qt6-base-dev libqt6gui6 libqt6widgets6 libqt6printsupport6 libqt6core6 libgl-dev libglx-dev libopengl-dev
```

### Fedora repository packages

_TODO: This is outdated advice._

The Qt, KDE and Valgrind build dependencies can be installed easily
by opening a terminal and issuing this command:

```
sudo yum install git gdb gcc-c++ qt-devel kdelibs-devel valgrind
```

\subsubsection{Windows (MSYS2)}

After installing [MSYS2](https://www.msys2.org), run this command in a MINGW64 shell:

```sh
pacman -S mingw-w64-clang-x86_64-qt6 cmake gcc make git
```

At the time of writing, this will use around 2Gb of disk space. Then continue to [build](#build).

\subsubsection{Windows (Without MinGW or MSYS2)}

If you have a development environment and for some reason want to use that over MSYS2 then ensure you run the installers for:

1. CMake: https://cmake.org/download/#latest
2. Qt: http://www.qt-project.org
3. A Text Editor for Code like Visual Studio Code: https://code.visualstudio.com/
4. A C compiler, like `gcc`, `cl`, `clang` or `tcc`.
5. Git Bash: https://gitforwindows.org/
6. A backend for CMake like Ninja: https://ninja-build.org/

Remember to add these to your `PATH` for scripts to use them.

This would give a similar build experience to standard development on Windows, but we recommend you use MSYS2.

Note that our behind-the-scenes Windows build uses Python to get the Qt libraries
[like this](https://github.com/Embroidermodder/libembroidery/blob/main/bin/build.sh).

\subsection{Build}

Assuming you have the dependencies for your system, on all systems with Bash, the following should work:

\begin{lstlisting}
bash build.sh
\end{lstlisting}

If your system does not have bash, it may still run as sh.
Failing that, try typing each line in in turn like this:

\begin{lstlisting}
git submodule init
git submodule update

cmake -S . -B"build" -G"Unix Makefiles" -DCMAKE_BUILD_TYPE="Debug"
cd build
cp -r ../assets/* .
cmake --build .
cat build.log
cd ..
\end{lstlisting}

\subsubsection{Running the Development Version}

After building as above, run your own development copy with:

\begin{lstlisting}
cd build
./embroidermodder2
\end{lstlisting}

\subsubsection{Troubleshooting}

If you have no luck with the above advice and still want to
run the development alpha, try reading the `build.log` in your
`build/` folder like this:

```sh
cat build/build.log
```

If, after googling keywords from the errors you're still stuck
post and issue on GitHub here: https://github.com/Embroidermodder/Embroidermodder/issues and supply the `build.log` file. If something
comes up a lot then we can add advice here.

## Development

During the alpha phase we mainly need to focus on getting the C bedrock of this project stable before letting more people
put their creations into it. In Beta, non-programming related contributions will be wecomed to the website and reference manual
repositories.

### Getting Involved

Anyone interested in changing Embroidermodder or becoming a contributor should go read our
[manuals](https://libembroidery.org/documentation), [make issues and submit patches](https://github.com/embroidermodder/refman)
as you find them because the project is very weak here. It will also serve as training for submitting patches to the actual
source code where changes are harder to critique and revise.

As for helping with specific bugs submitting an issue on GitHub along with the `debug-####.txt` file generated during
your test run would be the best approach. For longer term techniques see the next section.

### Bug Hunting

The Embroidermodder Project
===========================

.. toctree::
   :maxdepth: 2
   :caption: Contents
   :name: maintoc

   em_user
   mobile_user
   pet_user
   emrm
   credits

.. warning::

   IN ALPHA DEVELOPMENT: NOT READY FOR SERIOUS USE.

Embroidermodder is a free and open source machine embroidery application.
If our project is successful, it will:

\begin{itemize}
\item edit and create embroidery designs
\item estimate the amount of thread and machine time needed to stitch a design
\item convert embroidery files to a variety of formats
\item upscale or downscale designs
\item run on Windows, Mac and Linux
\end{itemize}

To try out the software in alpha see our `downloads page <https://libembroidery.org/downloads>`_.

Various sample embroidery design files can be found in
the github samples folder.

Screenshots
-----------

If you use multiple operating systems, it's important to choose
software that works on all of them. Embroidermodder 2 runs on Windows,
Linux and Mac OS X. Let's not forget the `Raspberry Pi <https://www.raspberrypi.org>`_.

\section{Documentation}

For all of these manuals (except the `embroider` manpage),
the source code is maintained as part
of the website [here](https://github.com/Embroidermodder/docs).

\subsection{The User Manuals}

Note that these URLs are maintained as the permalinks.

For all of these user manuals including the ``embroider`` manpage,
the source code is maintained as part
of the libembroidery project ([https://github.com/Embroidermodder/libembroidery](https://github.com/Embroidermodder/libembroidery)).
The documentation, like the code, is mostly common to all subprojects.

\begin{itemize}
\item Embroidermodder, EmbroideryMobile, PET: [https://www.libembroidery.org/user-manual](https://www.libembroidery.org/user-manual) ([PDF](https://www.libembroidery.org/em2_user_manual.pdf))
\item embroider: [https://www.libembroidery.org/embroider.txt](https://www.libembroidery.org/embroider.txt)
\end{itemize}

\subsection{The Developer Manual}

The Embroidermodder Reference Manual (EMRM) is the main developer 
resource found here: [https://www.libembroidery.org/refman](https://www.libembroidery.org/refman) with the printer friendly version here: [https://www.libembroidery.org/downloads/emrm.pdf](https://www.libembroidery.org/downloads/emrm.pdf).

\section{The Embroidermodder Project Website}

This directory contains most of the broader documentation and automation to
stop minor changes flooding each of Embroidermodder's sub-projects. Including
the documentation as webpages for the
[site itself](https://www.libembroidery.org), each subproject's user manual but
not the reference manual.

This frees the other repositories of the minor-commit heavy mundane tasks of
bundling software and separating a "user" build from a "production" build. It
also means that those projects aren't tasked with keeping production history.

For in-depth information about the software please read some of the PDF manual
included in the top level of the repository. Finishing the manual is the current
top priority in order to fascilitate new developers joining the project.

\subsection{Ideas}

A testing site that is maintained under testing.libembroidery.org so builds
don't go straight to the main landing page.

If this reaches the cap of storage offered by a github repository then we'll
have to think of something else since the version history of the binaries could
quickly become important if we have any regressions.

\section{Embroidermodder 2.0.0-alpha4  User Manual}

\subsection{Introduction}

!!! warning
    THIS MANUAL IS IN PROGRESS: PLEASE WAIT FOR THE BETA RELEASE.

This manual is for the various ways of using the \embname tools for designing,
editing and converting machine embroidery files. Most users should try loading and altering a design
in \embname itself before trying any of our conversion tools, our embedded system
or the command line interface. Advice on this is in the next section.

If you wish to write your own
software that uses these tools you will need the \embname Reference Manual (this
includes the API documentation). This is maintained at the permalink
\url{https://www.libembroidery.org/downloads/emrm.pdf}.

\begin{multicols}{2}
\footnotesize
\section{License}

\include{fdl-1.3.tex}
\end{multicols}

\normalsize
\bibliographystyle{plain}
\bibliography{references.bib}

\end{document}

